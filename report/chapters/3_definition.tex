\chapter{AlDiaCAR: System definition and phased development}

Having established the theoretical foundations and identified a distinct gap in the state-of-the-art of vehicle management applications in the preceding chapter, this chapter provides a comprehensive definition of the proposed solution: \textit{AlDiaCAR}. This chapter will serve as the architectural and functional blueprint for the project, articulating its core mission, foundational principles, and detailed feature set. The objective is to translate the conceptual framework into a tangible product specification that directly addresses the coordination frictions, maintenance blindness, and suboptimal decision-making prevalent in the modern multi-vehicle household, as exemplified by the Martínez family persona.

\textgap

The project's vision is ambitious, encompassing a wide range of functionalities from real-time coordination to intelligent, data-driven recommendations and behavioral reinforcement. To manage this complexity and ensure a methodologically sound development process, this chapter also introduces a phased implementation roadmap. This iterative approach deconstructs the complete vision into three distinct, sequential stages: the Alpha version, representing the core viable product; the Beta version, which enhances the system with intelligent features and user engagement mechanics; and finally, the Zenith release, which embodies the fully realized, feature-complete expression of the AlDiaCAR concept. This structured methodology allows for focused development, iterative testing, and the progressive validation of the project's hypotheses at each stage.

\section{The AlDiaCAR vision: A holistic framework}

The core mission of AlDiaCAR is to serve as the intelligent, collaborative "Digital garage" for the modern household. It is designed to seamlessly integrate vehicle management, user coordination, and sustainable decision-making into a single, cohesive, and effortless user experience. The system is architected upon four foundational pillars, each conceived to address a critical point of friction and inefficiency in the management of a shared, personal vehicle fleet. These pillars function interdependently to create a holistic solution that is more than the sum of its parts.

\subsection{Visual identity and design principles}

Before detailing the functional pillars, it is important to define the visual and interaction design philosophy that underpins the user experience. A consistent and thoughtful visual identity is critical for establishing user trust, ensuring usability, and creating an intuitive interface. The design of AlDiaCAR is guided by three core principles: clarity, accessibility, and encouragement. All visual elements, from the color palette to the typography, are selected to support these principles, creating a clean, modern, and non-intrusive user environment.

\textgap

Figure \ref{fig:visual-identity} presents the project's style guide, which specifies the primary and secondary color palettes, the chosen fonts for headings and body text, and the application logo. The color scheme utilizes a base of calming blues and greens to evoke a sense of reliability and sustainability, with vibrant accent colors used purposefully to draw attention to key actions and notifications.

\begin{figure}[H]
\centering
\includegraphics[width=0.9\textwidth]{images/branding/moodboard.png}
\caption{AlDiaCAR visual identity guide, defining the color palette, typography, and logo.}
\label{fig:visual-identity}
\end{figure}

\subsection{The four functional pillars}
The functionality of the AlDiaCAR system is structured around four distinct, yet interconnected, pillars. Each pillar is designed to address a specific set of pain points identified in the problem statement.

\subsubsection{Pillar 1: The shared digital key rack (Coordination)}
This foundational pillar is designed to definitively eradicate the persistent and friction-inducing "who has what car?" problem. It transforms the physical, opaque state of the household's vehicles into a transparent, digital, and universally accessible source of truth. Key functionalities include real-time vehicle status tracking, a seamless check-out/check-in protocol, and a proactive vehicle reservation system.

\subsubsection{Pillar 2: The proactive co-pilot (Intelligent recommendation)}
This pillar represents the core intelligence of the system, designed to answer the complex question, "What is the smartest vehicle choice for \textit{this specific trip}?" Its multi-vector analysis engine evaluates journeys against regulatory constraints (ZBEs), economic cost, environmental impact (CO\textsubscript{2} emissions), and maintenance awareness to provide a simple, ranked list of optimal choices.

\subsubsection{Pillar 3: The omniscient mechanic (Automated maintenance)}
This pillar acts as the household's collective, digital memory for vehicle health, ensuring that all automotive assets are consistently maintained. It combats "shared maintenance blindness" through smart maintenance profiles that track service intervals based on both time and mileage, and a shared alert system for upcoming tasks.

\subsubsection{Pillar 4: The household impact dashboard (Gamification \& analytics)}
This final pillar addresses the motivational aspects of sustainable mobility. It is designed to transform abstract data from daily use into tangible insights and positive reinforcement through a unified dashboard, personal and team-based achievements, and actionable analytics.

\section{Conceptual design and user flow wireframes}
To translate the abstract functional pillars into a concrete and user-centric interface, this section presents a series of low-fidelity wireframes. These conceptual sketches visualize the user experience of the Zenith release, illustrating the intended information architecture, component layout, and user flows for the application's most critical tasks. They serve as a visual blueprint that connects the system's defined features to its ultimate design goals.

\textgap

To translate the abstract functional pillars into a concrete and user-centric interface, this section presents a series of low-fidelity wireframes. These conceptual sketches visualize an initial, high-level vision for the user experience of the Zenith release, illustrating the potential information architecture, component layout, and user flows for the application's most critical tasks. This collection of early-stage designs, presented across Figure \ref{fig:sketches_part1} and Figure \ref{fig:sketches_part2}, serves as a visual blueprint that connects the system's defined features to its ultimate design goals.
\begin{figure}[H]
\centering
\includegraphics[width=\textwidth]{images/sketches/placeholder.png}
\label{fig:sketches_part1}
\end{figure}

\begin{figure}[H]
\centering
\includegraphics[width=\textwidth]{images/sketches/placeholder.png}
\label{fig:sketches_part2}
\end{figure}

\section{Phased implementation roadmap}

To systematically construct the comprehensive system described above, a phased development roadmap is proposed. This iterative methodology deconstructs the full feature set into three manageable and logically sequential versions. Each phase builds upon the last, allowing for focused development, targeted testing, and the potential for user feedback to inform subsequent stages. The three phases are designated as the Alpha version, the Beta version, and the Zenith release.

\subsection{The Alpha version: Establishing the core viable product}
The primary objective of the Alpha phase is to build and validate the foundational functionality of AlDiaCAR. This version is conceived as the Minimum Viable Product (MVP), focusing exclusively on solving the most acute and immediate pain points: coordination friction and shared maintenance blindness.
\textgap

\textbf{Features of the Alpha version:}
\begin{itemize}
\item \textbf{Pillar 1 (Coordination):}
\begin{itemize}
\item User and multiple vehicle registration.
\item A simple, clear dashboard showing the status of each vehicle.
\item Manual, one-tap "Check-Out" and "Check-In" functionality to update a vehicle's status between At Home and In Use. The user who checked out the vehicle is displayed.
\end{itemize}
\item \textbf{Pillar 2 (Recommendation):} This pillar is \textbf{not implemented} in the Alpha version.
\item \textbf{Pillar 3 (Maintenance):}
\begin{itemize}
\item A dedicated section for each vehicle to manually log key maintenance dates.
\item Simple, time-based push notifications when a logged date is approaching.
\end{itemize}
\item \textbf{Pillar 4 (Gamification):} This pillar is \textbf{not implemented} in the Alpha version.
\end{itemize}
The Alpha version is a digital replacement for the family's whiteboard, validating the core need for a centralized coordination hub.

\subsection{The Beta version: Enhancing intelligence and engagement}
Building upon the stable foundation of the Alpha version, the Beta phase introduces the "smart" features that define AlDiaCAR's unique value proposition.
\textgap

\textbf{Features of the Beta version (Includes all Alpha features, plus):}
\begin{itemize}
\item \textbf{Pillar 1 (Coordination):} Introduction of the vehicle \textbf{Reservation System}.
\item \textbf{Pillar 2 (Recommendation):} Implementation of an \textbf{initial Recommendation Engine} limited to \textbf{ZBE constraint analysis}.
\item \textbf{Pillar 3 (Maintenance):} Enhanced tracking with \textbf{mileage-based intervals}.
\item \textbf{Pillar 4 (Gamification):} A \textbf{first layer of gamification} with individual achievements and badges.
\end{itemize}
The Beta version transforms AlDiaCAR from a simple utility into an intelligent assistant.

\subsection{The Zenith release: The complete vision}
The Zenith release represents the culmination of the development roadmap, a feature-complete version of the application that fully realizes the initial four-pillar vision.
\textgap

\textbf{Features of the Zenith release (Includes all Beta features, plus):}
\begin{itemize}
\item \textbf{Pillar 1 (Coordination):} Polished UI/UX with potential integration into third-party calendar applications.
\item \textbf{Pillar 2 (Recommendation):} The \textbf{full, multi-factor Recommendation Engine} with cost, CO\textsubscript{2}, and maintenance awareness.
\item \textbf{Pillar 3 (Maintenance):} A shared, collaborative \textbf{Household Task List} for maintenance items.
\item \textbf{Pillar 4 (Gamification):} The complete \textbf{Household Impact Dashboard} with advanced analytics and collaborative goals.
\end{itemize}
The Zenith release is the ultimate expression of AlDiaCAR: an intelligent layer that removes friction, saves money, enhances safety, and systematically guides the family toward more sustainable mobility.
