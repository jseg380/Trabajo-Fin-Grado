\chapter{Introduction}

\section{Context and relevance}

According to a 2020 study, transportation accounted for 16.2\% of global greenhouse gas emissions in 2016, with road transport contributing 11.9\% of total emissions. A significant 60\% of road transport emissions originate from passenger travel\cite{owid-ghg-emissions-by-sector}. This is largely due to car dependency in suburban and rural areas, where fragmented public transit systems lead to inefficient vehicle usage patterns. Recent 2022 data analysis reveals a 14.3\% increase in passenger cars per thousand inhabitants across the EU between 2012-2022\cite{passengers-cars-per-thousand-people}. Consequently, individuals increasingly own multiple vehicles with varying emissions profiles, frequently resulting in suboptimal transportation choices for daily trips.

The implications of this trend are twofold:
\begin{itemize}
    \item \textbf{Environmental impact}: Private vehicles typically demonstrate lower efficiency than public transport alternatives, generating higher emissions per passenger-kilometer.
    
    \item \textbf{Operational complexity}: Individuals with multiple vehicles face challenges in coordinating maintenance schedules and optimizing usage efficiency.
\end{itemize}

Sustainable private vehicle usage requires minimizing emissions through:
\begin{itemize}
    \item Preventive maintenance to ensure optimal engine performance
    
    \item Conscious vehicle selection based on ecological impact and trip requirements
    
    \item Behavioral interventions promoting eco-driving habits
\end{itemize}

Green computing principles significantly enhance this approach. Software solutions capable of influencing user behavior can achieve substantial emissions reductions at scale without requiring hardware modifications. This project therefore aligns with ``Green Algorithms'' initiatives by transforming individual decisions into environmental benefits through digital tools.

\section{Problem statement and user persona}

\subsection{Core problem}

Individuals managing multiple personal vehicles lack integrated tools to:

\begin{itemize}
    \item Centralize maintenance schedules (ITV inspections, tire replacements) across their fleet

    \item Optimize vehicle selection based on sustainability metrics

    \item Track and incentivize eco-conscious usage patterns
\end{itemize}

\subsection{User persona}
\textbf{Alejandro Martínez}

\vspace{0.7em}

\begin{flushleft}
    Scenario: Alejandro lives in a household with three vehicles (2015 diesel SUV, 2020 hybrid sedan, 2022 electric hatchback). He struggles to manage maintenance schedules and often selects vehicles suboptimally for trips.
\end{flushleft}

Pain Points:

\begin{itemize}
    \item Maintenance Overlooks: Missed the SUV's oil change deadline, causing engine inefficiency and premature wear of components
    
    \item Suboptimal Selection: Used the SUV for a 18km urban trip despite the electric car being available
    
    \item Usage Tracking: No centralized system to compare trip emissions across vehicles
    
    \item Eco-Awareness Gap: Lacks feedback on how vehicle choices impact carbon footprint
\end{itemize}

This results in higher operational costs, accelerated vehicle wear, and avoidable emissions – challenges addressable through personal fleet management.

\section{Objectives and scope}

\subsection{Primary objective}

Design and develop a cross-platform application (Android/web/iOS) to optimize personal vehicle usage for individuals managing multiple vehicles through:

\begin{itemize}
    \item Centralized vehicle management
    
    \item Sustainability-driven recommendations
    
    \item Behavioral reinforcement mechanisms
\end{itemize}

\subsection{Specific objectives}

\begin{table}[h]
    \centering
    \begin{tabular}{@{}p{0.35\textwidth} p{0.6\textwidth}@{}}
        \toprule
        \textbf{Objective} & \textbf{Technical Approach} \\ \midrule
        1. Cross-platform accessibility & Develop using React Native (Expo) for universal device access \\
        2. Maintenance automation & Implement mileage/time-based triggers (oil changes, tire rotations) \\
        3. Vehicle selection optimization & Create heuristic algorithms recommending optimal vehicle per trip \\
        4. Emissions tracking & Calculate CO\textsubscript{2} output using vehicle-specific emission factors \\
        5. Gamification system & Design achievement badges for sustainable usage patterns \\ \bottomrule
    \end{tabular}
    \caption{Project objectives and technical approaches}
\end{table}

\subsection{Scope boundaries}

Included:

\begin{itemize}
    \item Private vehicles (cars) managed by individual users
    
    \item Support for multiple vehicles per user
    
    \item Rule-based recommendation heuristics
    
    \item Third-party technical car specs APIs (Auto-data API)
\end{itemize}

Excluded:

\begin{itemize}
    \item Multi-user coordination features
    
    \item Public transport integration
    
    \item Machine learning components
\end{itemize}

\section{Methodology overview}

\textbf{Development approach}

\begin{itemize}
    \item Cross-platform strategy:
    \begin{itemize}
        \item Frontend: Expo (React Native) for reusable UI components
        \item Backend: Express.js (Node.js) REST API with MongoDB
    \end{itemize}
    
    \item Testing protocol:
    \begin{itemize}
        \item Unit testing (Jest) for backend controllers
        \item End-to-end testing (Playwright) for core user journeys
    \end{itemize}
    
    \item Deployment:
    \begin{itemize}
        \item Docker containers for environment consistency
    \end{itemize}
\end{itemize}
