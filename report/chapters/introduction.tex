\chapter{Introduction}

\section{Context and relevance: From global emissions to household mobility}

The escalating climate crisis represents one of the most significant challenges of the 21st century, demanding urgent and innovative solutions across all sectors of society. Transportation stands out as a critical area for intervention. According to comprehensive data analysis, the transport sector was responsible for approximately 16.2\% of global greenhouse gas (GHG) emissions in 2016. Within this figure, road transport—comprising cars, motorcycles, buses, and trucks—was the largest contributor, accounting for 11.9\% of total global emissions. A striking 60\% of these road transport emissions originate from passenger travel alone, primarily from the use of private vehicles \cite{owid-ghg-emissions-by-sector}.

\textgap

This reliance on private cars is particularly pronounced in developed nations. In the European Union, for instance, the number of passenger cars per thousand inhabitants grew by 14.3\% between 2012 and 2022, signaling a persistent trend towards private vehicle ownership \cite{passengers-cars-per-thousand-people}. This trend is exacerbated in suburban and rural landscapes, where fragmented or insufficient public transit systems often make private vehicles a necessity rather than a choice. The result is a modern household dynamic defined by multiple vehicles, each with a distinct age, fuel type, and emissions profile. This complexity introduces significant inefficiencies, both economically and environmentally, as daily transportation choices are often made based on convenience or availability rather than optimization.

\textgap

This thesis addresses the intersection of these trends by focusing on sustainability in private vehicle use. The concept extends beyond simply owning an electric vehicle; it encompasses a holistic approach that includes diligent maintenance to ensure peak operational efficiency, conscious vehicle selection for each specific journey, and the adoption of eco-driving habits.

\textgap

Furthermore, this project is grounded in the principles of Green Computing, also known as Green Information and Communication Technology (ICT). While often associated with reducing the energy consumption of data centers and hardware, a crucial pillar of Green Computing is "ICT for Sustainability"—the application of software to influence and improve real-world processes. By developing intelligent software solutions, we can empower users to make more informed, environmentally conscious decisions at a scale unachievable through manual methods. This project posits that a well-designed mobile application can serve as a persuasive technology, nudging user behavior towards sustainable mobility patterns and transforming individual choices into a collective, positive environmental impact.

\section{The problem statement: the coordination gap in the modern multi-driver household}

\subsection{Core problem}

Individuals managing multiple personal vehicles lack integrated tools to:

\begin{itemize}
    \item Centralize maintenance schedules (ITV inspections, tire replacements) across their fleet

    \item Optimize vehicle selection based on sustainability metrics

    \item Track and incentivize eco-conscious usage patterns
\end{itemize}

\subsection{User persona}
\textbf{Alejandro Martínez}

\vspace{0.7em}

\begin{flushleft}
    Scenario: Alejandro lives in a household with three vehicles (2015 diesel SUV, 2020 hybrid sedan, 2022 electric hatchback). He struggles to manage maintenance schedules and often selects vehicles suboptimally for trips.
\end{flushleft}

Pain Points:

\begin{itemize}
    \item Maintenance Overlooks: Missed the SUV's oil change deadline, causing engine inefficiency and premature wear of components
    
    \item Suboptimal Selection: Used the SUV for a 18km urban trip despite the electric car being available
    
    \item Usage Tracking: No centralized system to compare trip emissions across vehicles
    
    \item Eco-Awareness Gap: Lacks feedback on how vehicle choices impact carbon footprint
\end{itemize}

This results in higher operational costs, accelerated vehicle wear, and avoidable emissions – challenges addressable through personal fleet management.

\section{Objectives and scope}

\subsection{Primary objective}

Design and develop a cross-platform application (Android/web/iOS) to optimize personal vehicle usage for individuals managing multiple vehicles through:

\begin{itemize}
    \item Centralized vehicle management
    
    \item Sustainability-driven recommendations
    
    \item Behavioral reinforcement mechanisms
\end{itemize}

\subsection{Specific objectives}

\begin{table}[h]
    \centering
    \begin{tabular}{@{}p{0.35\textwidth} p{0.6\textwidth}@{}}
        \toprule
        \textbf{Objective} & \textbf{Technical Approach} \\ \midrule
        1. Cross-platform accessibility & Develop using React Native (Expo) for universal device access \\
        2. Maintenance automation & Implement mileage/time-based triggers (oil changes, tire rotations) \\
        3. Vehicle selection optimization & Create heuristic algorithms recommending optimal vehicle per trip \\
        4. Emissions tracking & Calculate CO\textsubscript{2} output using vehicle-specific emission factors \\
        5. Gamification system & Design achievement badges for sustainable usage patterns \\ \bottomrule
    \end{tabular}
    \caption{Project objectives and technical approaches}
\end{table}

\subsection{Scope boundaries}

Included:

\begin{itemize}
    \item Private vehicles (cars) managed by individual users
    
    \item Support for multiple vehicles per user
    
    \item Rule-based recommendation heuristics
    
    \item Third-party technical car specs APIs (Auto-data API)
\end{itemize}

Excluded:

\begin{itemize}
    \item Multi-user coordination features
    
    \item Public transport integration
    
    \item Machine learning components
\end{itemize}

\section{Methodology overview}

\textbf{Development approach}

\begin{itemize}
    \item Cross-platform strategy:
    \begin{itemize}
        \item Frontend: Expo (React Native) for reusable UI components
        \item Backend: Express.js (Node.js) REST API with MongoDB
    \end{itemize}
    
    \item Testing protocol:
    \begin{itemize}
        \item Unit testing (Jest) for backend controllers
        \item End-to-end testing (Playwright) for core user journeys
    \end{itemize}
    
    \item Deployment:
    \begin{itemize}
        \item Docker containers for environment consistency
    \end{itemize}
\end{itemize}
