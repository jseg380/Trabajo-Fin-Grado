\chapter{Introduction}

\section{Context and relevance: global emissions and personal vehicle management}

The escalating climate crisis represents one of the most significant challenges of the 21st century, demanding urgent and innovative solutions across all sectors of society. Transportation stands out as a critical area for intervention. According to comprehensive data analysis, the transport sector was responsible for approximately 16.2\% of global greenhouse gas (GHG) emissions in 2016. Within this figure, road transport—comprising cars, motorcycles, buses, and trucks—was the largest contributor, accounting for 11.9\% of total global emissions. A striking 60\% of these road transport emissions originate from passenger travel alone, primarily from the use of private vehicles \cite{owid-ghg-emissions-by-sector}.

\begin{figure}[H]
    \centering
    \includegraphics[width=0.55\textwidth]{images/ghg-emissions-by-sector.png}
    \caption{Breakdown of global greenhouse gas emissions by sector.}
\end{figure}

This reliance on private cars is particularly pronounced in developed nations. In the European Union, for instance, the number of passenger cars per thousand inhabitants reached 560 in 2022, signaling a persistent trend towards private vehicle ownership \cite{passengers-cars-per-thousand-people}. This trend, combined with longer vehicle lifecycles, presents a new challenge for the individual owner: managing a personal fleet that often consists of multiple vehicles, each with a distinct age, fuel type, and emissions profile. This complexity introduces significant inefficiencies, as daily transportation choices are often made based on convenience rather than optimal, sustainable selection.

\textgap

This thesis addresses these inefficiencies by focusing on holistic sustainability in private vehicle use. The concept extends beyond simply owning an electric vehicle; it encompasses diligent maintenance to ensure peak operational efficiency, as poorly maintained vehicles can see a significant increase in fuel consumption and pollutant emissions \cite{iea2021fuel}. It also involves conscious vehicle selection for each journey and the adoption of eco-driving habits.

\textgap

Furthermore, this project is grounded in the principles of Green Computing, specifically "ICT for Sustainability." This pillar of Green IT focuses on applying software to influence and improve real-world processes. By developing intelligent software solutions, we can empower users to make more informed, environmentally conscious decisions. This project posits that a well-designed mobile application can serve as a persuasive technology, a concept defined as technology intentionally designed to change a person's attitudes or behaviors \cite{fogg2002persuasive}, thereby nudging users towards sustainable mobility patterns.

\section{The problem statement: the management gap for the modern vehicle owner}

\subsection{Core problem}
Individuals managing multiple personal vehicles lack integrated tools to:
\begin{itemize}
    \item Centralize maintenance schedules (e.g., technical inspections, oil changes) across their fleet.
    \item Optimize vehicle selection for trips based on sustainability metrics.
    \item Track and be incentivized for eco-conscious usage patterns.
\end{itemize}

\subsection{User persona}
\textbf{Alejandro Martínez}
\vspace{0.7em}
\begin{flushleft}
    \textbf{Scenario:} Alejandro is a tech-savvy professional who owns three vehicles: a 2015 diesel SUV, a 2020 hybrid sedan, and a 2022 electric hatchback. As the primary manager of this personal fleet, he struggles to keep track of the different maintenance needs and timelines for each car and often makes suboptimal choices for short trips out of habit.
\end{flushleft}

\textbf{Pain points:}
\begin{itemize}
    \item \textbf{Maintenance Overlooks:} He recently missed the SUV's oil change deadline, leading to reduced engine efficiency and potential for premature component wear.
    \item \textbf{Suboptimal Selection:} He used the high-emission SUV for a 18km urban trip despite the fully charged electric car being available, simply because it was parked more conveniently.
    \item \textbf{Usage Tracking:} He has no centralized system to compare the total trip emissions or running costs across his different vehicles.
    \item \textbf{Eco-Awareness Gap:} He lacks clear, actionable feedback on how his daily vehicle choices impact his carbon footprint.
\end{itemize}

This results in higher operational costs, accelerated vehicle wear, and avoidable emissions—challenges directly addressable through a personal fleet management application, as illustrated in Figure \ref{fig:user-problem-diagram}.

\begin{figure}[h!]
    \centering
    \includegraphics[width=0.8\textwidth]{images/user-problem-diagram.png}
    \caption{Conceptual diagram of the user's suboptimal vehicle selection problem.}
    \label{fig:user-problem-diagram}
\end{figure}


\section{Objectives and scope}

\subsection{Primary objective}
Design and develop a cross-platform application (Android/web/iOS) to help an individual optimize the use of their multiple vehicles through:
\begin{itemize}
    \item Centralized vehicle and maintenance management.
    \item Sustainability-driven trip recommendations.
    \item Behavioral reinforcement via gamification.
\end{itemize}

\subsection{Specific objectives}
\begin{table}[h]
    \centering
    \caption{Project objectives and technical approaches}
    \begin{tabular}{@{}p{0.35\textwidth} p{0.6\textwidth}@{}}
        \toprule
        \textbf{Objective} & \textbf{Technical approach} \\ \midrule
        1. Cross-platform accessibility & Develop using React Native (Expo) for universal device access. \\
        2. Maintenance automation & Implement alerts based on time and mileage triggers. \\
        3. Vehicle selection optimization & Create a heuristic algorithm to recommend the optimal vehicle per trip based on CO\textsubscript{2} output. \\
        4. Emissions tracking & Calculate CO\textsubscript{2} output using vehicle-specific emission factors. \\
        5. Gamification system & Design an achievement and badge system for sustainable usage. \\ \bottomrule
    \end{tabular}
\end{table}

\subsection{Scope boundaries}
\textbf{Included:}
\begin{itemize}
    \item Private vehicles (cars) managed by a single user account.
    \item Support for multiple vehicles per user.
    \item Rule-based recommendation heuristics.
    \item Integration with a mock third-party API for technical vehicle specifications.
\end{itemize}

\textbf{Excluded:}
\begin{itemize}
    \item Multi-user/household coordination features.
    \item Real-time traffic data or public transport integration.
    \item Machine learning components for predictive analysis.
\end{itemize}

\section{Methodology overview}
\textbf{Development approach:}
\begin{itemize}
    \item \textbf{Cross-platform strategy:}
    \begin{itemize}
        \item Frontend: Expo (React Native) for reusable UI components.
        \item Backend: Express.js (Node.js) REST API with MongoDB.
    \end{itemize}
    \item \textbf{Testing protocol:}
    \begin{itemize}
        \item Backend Testing: Unit and integration tests using Jest.
        \item Frontend Testing: End-to-end tests for core user journeys using Playwright.
    \end{itemize}
    \item \textbf{Deployment environment:}
    \begin{itemize}
        \item Docker and Docker Compose for a consistent and reproducible development environment.
    \end{itemize}
\end{itemize}
