\chapter{System Design and Architecture}

This chapter details the architectural design of the application, named \textit{AlDiaCAR}\footnote{A placeholder name for the application, derived from "Al Día" (up-to-date) and "Car".}. The design is guided by the project's core objectives: to provide a cross-platform, scalable, and maintainable solution for optimizing household vehicle use with a focus on sustainability. We will discuss the high-level architecture, break down the system into its primary components, define the data flow and API contracts, justify the technology stack, and outline the logic for the sustainability and gamification features.

\section{High-Level Architecture}

The system is designed following a classic client-server architecture, which decouples the user interface from the core business logic and data storage. This model enhances security, scalability, and allows for multiple types of clients to be developed in the future (e.g., a web application) without altering the backend.

The main components are:
\begin{itemize}
    \item \textbf{Frontend Client:} A cross-platform mobile application built with React Native and the Expo framework. It is responsible for all user interactions, data presentation, and communication with the backend via a REST API.
    \item \textbf{Backend Server:} A Node.js application using the Express.js framework. It serves as the central hub, handling business logic, user authentication, data processing, and communication with the database.
    \item \textbf{Database:} A MongoDB NoSQL database, which stores all persistent data, including user profiles, vehicle details, trip logs, and gamification state.
    \item \textbf{External APIs (Future Integration):} The architecture is designed to integrate with third-party APIs for enriching vehicle data (e.g., technical specifications, maintenance schedules) and potentially real-time route information.
\end{itemize}

Figure \ref{fig:high-level-arch} provides a visual representation of this architecture, illustrating the flow of information between the components.

\begin{figure}[h!]
    \centering
    % Placeholder for architecture diagram.
    % Example: \includegraphics[width=0.9\textwidth]{images/architecture-diagram.png}
    \framebox[0.9\textwidth][c]{
        \parbox{0.85\textwidth}{
            \vspace{4cm}
            \centering
            \textbf{Figure Placeholder:} A diagram showing a mobile device (Frontend) connecting via REST API to a server block (Backend: Node.js/Express). The server connects to a database cylinder (MongoDB). An arrow points from "External APIs" towards the Backend.
            \vspace{4cm}
        }
    }
    \caption{High-Level System Architecture}
    \label{fig:high-level-arch}
\end{figure}

\section{Component Overview}

The system is logically divided into frontend and backend components, each with distinct responsibilities and internal structures that reflect the project's source code organization.

\subsection{Frontend Components (React Native)}
The frontend is structured as a mobile application using Expo, with a file-based routing system. The key components, located in `src/frontend/app/`, are:

\begin{itemize}
    \item \textbf{Authentication (`/app/(auth)`):} Contains screens for user registration (`register.tsx`) and login (`login.tsx`). It handles user input, form validation, and communicates with the backend's authentication endpoints.
    \item \textbf{Main Application Tabs (`/app/(tabs)`):} This is the core user-facing part of the app, organized into several tabs:
    \begin{itemize}
        \item \textbf{Home (`/home`):} The main dashboard, designed to show a summary of important information, such as upcoming maintenance reminders, the last trip taken, and a quick recommendation for the most sustainable vehicle available.
        \item \textbf{Vehicles (`/vehicles`):} A module for managing the household's vehicles. Users can view a list of all registered vehicles, add new ones, and edit their details (e.g., make, model, year, fuel type).
        \item \textbf{Routes (`/routes`):} The interface for the recommendation system. Users can input a destination to receive a suggestion on which vehicle to use, based on sustainability metrics.
        \item \textbf{Statistics (`/stats`):} This screen implements the gamification and data visualization aspects, showing user statistics, earned badges, and sustainability scores.
    \end{itemize}
    \item \textbf{User Profile (`/app/profile`):} Displays user information and provides access to application settings (`settings.tsx`).
    \item \textbf{State Management (`/context/AuthContext.tsx`):} A React Context is used to manage global application state, primarily for user authentication, ensuring that user session data is available throughout the component tree.
    \item \textbf{Localization (`/localization`):} The application supports internationalization with dedicated folders for English (`en-UK`) and Spanish (`es-ES`), allowing for easy translation of all user-facing text.
\end{itemize}

\subsection{Backend Components (Node.js)}
The backend, located in `src/backend/`, is a RESTful API built with Node.js and Express.js. Its structure promotes separation of concerns:

\begin{itemize}
    \item \textbf{Routes (`/routes`):} Defines the API endpoints. For instance, `authRoutes.js` handles `/api/auth/register` and `/api/auth/login`, while `userRoutes.js` (a placeholder for vehicle and trip routes) would manage vehicle and trip-related endpoints.
    \item \textbf{Controllers (`/controllers`):} Contains the business logic for each route. For example, `authController.js` implements the logic for user registration, password hashing, and token generation. `userController.js` will manage fetching and updating user and vehicle data.
    \item \textbf{Models (`/models`):} Defines the data schemas for the MongoDB database using the Mongoose library. The core models are:
    \begin{itemize}
        \item \textbf{User.js:} Schema for user accounts, including name, email, hashed password, and associations with vehicles and trips.
        \item \textbf{Vehicle.js:} Schema for vehicles, storing properties like make, model, registration year, fuel type, user-defined emission factor, and maintenance schedules.
        \item \textbf{Trip.js:} Schema for trip logs, linking a user and a vehicle to a specific journey, and storing data like distance, duration, and calculated CO$_2$ emissions.
    \end{itemize}
    \item \textbf{Middleware (`/middleware`):} Contains functions that execute before the main route handlers. `authMiddleware.js` is crucial for protecting routes by verifying the user's JSON Web Token (JWT).
    \item \textbf{Database Configuration (`/config/db.js`):} Manages the connection to the MongoDB database, using environment variables for security.
\end{itemize}

\subsection{Data Model}
The NoSQL data model, implemented in MongoDB, was chosen for its flexibility. The primary collections are \texttt{Users}, \texttt{Vehicles}, and \texttt{Trips}. Relationships are managed through object references (e.g., a `Trip` document contains a reference to the `\_id` of the `User` and `Vehicle`). This structure is well-suited for the application's needs, as vehicle attributes and maintenance requirements can vary significantly.

% Placeholder for a simple ERD or UML class diagram
% \begin{figure}[h!]
%    \centering
%    \includegraphics[width=0.8\textwidth]{images/data-model-diagram.png}
%    \caption{Simplified Data Model (ERD)}
%    \label{fig:data-model}
% \end{figure}

\section{Data Flow and API Design}
Communication between the frontend and backend occurs over HTTPS via a RESTful API. The API uses standard HTTP methods (GET, POST, PUT, DELETE) and JSON for data interchange.

A typical data flow for a core feature, such as getting a vehicle recommendation, is as follows:
\begin{enumerate}
    \item The user enters a destination in the frontend's "Routes" screen and requests a recommendation.
    \item The React Native client sends a POST request to a backend endpoint, such as `/api/recommendations/`, including the route details and the user's ID in the authenticated request.
    \item The backend's authentication middleware verifies the user's JWT.
    \item The recommendation controller fetches all vehicles associated with the user's household from the database.
    \item The controller's algorithm processes each vehicle, calculating the estimated emissions for the proposed route based on the vehicle's stored emission factor.
    \item The backend responds with a sorted list of vehicles, with the most sustainable option first, in a JSON format.
    \item The frontend parses the JSON response and displays the recommendation to the user.
\end{enumerate}

Key API endpoints designed for the system include:
\begin{verbatim}
-- Authentication --
POST   /api/auth/register
POST   /api/auth/login

-- Vehicle Management --
GET    /api/vehicles
POST   /api/vehicles
GET    /api/vehicles/:id
PUT    /api/vehicles/:id
DELETE /api/vehicles/:id

-- Trip Management --
POST   /api/trips/start
POST   /api/trips/end
GET    /api/trips?userId=<id>

-- Recommendation --
POST   /api/recommendations
\end{verbatim}

\section{Non-Functional Requirements}
Beyond the core features, the system was designed with several non-functional requirements in mind:
\begin{itemize}
    \item \textbf{Cross-Platform Compatibility:} The choice of React Native ensures that a single codebase can be deployed to both iOS and Android, reducing development and maintenance effort.
    \item \textbf{Scalability:} The backend is stateless, allowing for horizontal scaling by running multiple instances behind a load balancer. MongoDB also offers robust scaling capabilities.
    \item \textbf{Security:} User authentication is handled using JSON Web Tokens (JWTs). Passwords are not stored in plain text; they are hashed using bcrypt. All communication between client and server is intended to be over HTTPS in a production environment.
    \item \textbf{Maintainability:} The project is structured into logical modules with a clear separation of concerns (models, views, controllers on the backend; components, screens, services on the frontend). This modularity, combined with the use of TypeScript in the frontend, improves code quality and makes future maintenance easier.
    \item \textbf{Usability:} The application provides internationalization support and aims for a clean, intuitive user interface to encourage adoption.
\end{itemize}

\section{Technology Choices Justification}
The technology stack was selected to align with the project requirements and leverage modern, well-supported tools.
\begin{itemize}
    \item \textbf{React Native (with Expo):} Chosen for its ability to build native-quality applications for iOS and Android from a single JavaScript/TypeScript codebase. Expo further accelerates development by simplifying the build process, providing a rich library of pre-built components, and managing native project configuration.
    \item \textbf{Node.js with Express.js:} Selected for the backend due to its high performance with asynchronous, non-blocking I/O, making it ideal for an API-driven application. The vast npm ecosystem and the fact that it uses JavaScript create language synergy with the frontend.
    \item \textbf{MongoDB:} A NoSQL database was preferred over a relational one due to its flexible schema. This is advantageous for storing vehicle data, where different cars may have different attributes and maintenance items. Its document-based structure maps naturally to JavaScript objects.
    \item \textbf{Docker:} The inclusion of a `docker-compose.yaml` file facilitates a consistent and isolated development environment. It allows any developer to run the backend and database with a single command, eliminating "it works on my machine" issues.
    \item \textbf{Jest \& Playwright:} These frameworks were selected for the testing strategy. Jest is a standard for unit testing JavaScript/Node.js applications, while Playwright is a powerful tool for end-to-end testing that can simulate user interactions across different platforms.
\end{itemize}

\section{Gamification and Sustainability Logic}
The sustainability and gamification features are central to the application's goal of influencing user behavior. The logic for these features is implemented on the backend.
\begin{itemize}
    \item \textbf{Sustainability Recommendation:} The initial recommendation algorithm is heuristic-based. It calculates the total CO$_2$ emissions for a given trip using a simplified formula:
    \begin{equation}
        \text{Emissions (gCO$_2$)} = \text{Distance (km)} \times \text{Emission Factor (gCO$_2$/km)}
    \end{equation}
    The `Emission Factor` is a value stored for each vehicle, which can be sourced from official manufacturer data or entered by the user. The system recommends the vehicle with the lowest calculated emissions for the trip.
    \item \textbf{Gamification:} The gamification module is designed to reward sustainable behavior and consistent maintenance. The backend will track user actions and award badges or points based on predefined rules, such as:
    \begin{itemize}
        \item Completing a maintenance task on time.
        \item Choosing the recommended sustainable vehicle for a trip.
        \item Achieving a certain number of "green" trips in a month.
        \item Reducing the household's average emissions over a period.
    \end{itemize}
    These achievements will be stored in the database and displayed on the user's statistics screen to provide positive reinforcement.
\end{itemize}

This architectural foundation provides a robust and extensible platform for developing the \textit{AlDiaCAR} application, balancing rapid development with long-term maintainability and scalability.
