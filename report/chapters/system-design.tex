\chapter{System design and architecture}

This chapter details the architectural design of the application, named \textit{AlDiaCAR}\footnote{A placeholder name for the application, derived from "Al Día" (up-to-date) and "Car".}. The design is guided by the project's core objectives: to provide a cross-platform, scalable, and maintainable solution for optimizing an individual's personal vehicle use with a focus on sustainability. We will discuss the high-level architecture, break down the system into its primary components, define the data flow, justify the technology stack, and outline the logic for the sustainability and gamification features.

\section{High-level architecture}

The system is designed following a classic client-server architecture, which decouples the user interface from the core business logic and data storage. This model enhances security, scalability, and allows for multiple types of clients to be developed in the future without altering the backend.

The main components are:
\begin{itemize}
    \item \textbf{Frontend client:} A cross-platform mobile application built with React Native and the Expo framework. It is responsible for all user interactions, data presentation, and communication with the backend.
    \item \textbf{Backend server:} A Node.js application using the Express.js framework. It serves as the central hub, handling business logic, user authentication, data processing, and communication with the database.
    \item \textbf{Database:} A MongoDB NoSQL database that stores all persistent data, including user profiles, vehicle details, and trip logs.
    \item \textbf{Mock API:} A secondary Node.js server that simulates a third-party API for retrieving vehicle specifications, ensuring development and testing can occur offline and with deterministic data.
\end{itemize}

Figure \ref{fig:high-level-arch} provides a visual representation of this architecture and the data flow between components within the development environment.

\begin{figure}[h!]
    \centering
    \begin{tikzpicture}[
        node distance=2cm and 2.5cm,
    ]
    % Components
    \node[user] (user) {User};
    \node[comp, right=of user] (frontend) {\textbf{Frontend Client} \\ React Native (Expo)};
    
    % Backend server, placed below frontend
    \node[comp, below=of frontend, yshift=-1cm] (backend) {\textbf{Backend API} \\ Node.js / Express.js};
    
    % Mock API server
    \node[comp, right=of backend] (mockapi) {\textbf{Mock Car API} \\ Node.js};

    % Database
    \node[db, left=of backend] (db) {MongoDB};

    % Docker environment container
    \begin{pgfonlayer}{background}
        \node[env={Docker Environment}, fit=(backend) (db), inner sep=0.5cm] (dockerenv) {};
    \end{pgfonlayer}

    % Arrows
    \draw[arrow] (user.east) -- (frontend.west);
    \draw[arrow] (frontend.south) -- (backend.north) node[midway, right, xshift=0.1cm] {REST API};
    \draw[arrow] (backend.west) -- (db.east) node[midway, below] {Mongoose};
    \draw[arrow, dashed] (frontend.east) .. controls +(east:1) and +(north:1.5) .. (mockapi.north) node[midway, above, sloped] {Mock API Call};

    \end{tikzpicture}
    \caption{High-level system architecture, including the development environment.}
    \label{fig:high-level-arch}
\end{figure}


\section{Component overview}

The system is logically divided into frontend and backend components, each with distinct responsibilities that reflect the project's source code organization.

\subsection{Frontend components (React Native)}
The frontend is structured as a mobile application using Expo, with a file-based routing system. The key components, located in \texttt{src/frontend/app/}, are:
\begin{itemize}
    \item \textbf{Authentication (`/app/(auth)`):} Screens for user registration and login.
    \item \textbf{Main application tabs (`/app/(tabs)`):} The core user-facing part of the app:
    \begin{itemize}
        \item \textbf{Home (`/home`):} A dashboard showing upcoming maintenance alerts.
        \item \textbf{Vehicles (`/vehicles`):} A module for a user to manage their personal fleet, including adding, viewing, and editing vehicles.
        \item \textbf{Routes (`/routes`):} The interface for the recommendation system.
        \item \textbf{Statistics (`/stats`):} Implements gamification and data visualization, showing user stats and earned achievements.
    \end{itemize}
    \item \textbf{State management (`/context/AuthContext.tsx`):} A React Context that manages the global authentication state, making user session data available throughout the component tree.
    \item \textbf{Localization (`/localization`):} Supports internationalization with dedicated folders for English and Spanish, allowing for easy translation of all UI text.
\end{itemize}

\subsection{Backend components (Node.js)}
The backend, located in \texttt{src/backend/}, is a RESTful API built with Node.js and Express.js. Its structure promotes separation of concerns:
\begin{itemize}
    \item \textbf{Routes (`/routes`):} Defines the API endpoints. The project uses dedicated files for each resource (e.g., \texttt{authRoutes.js}, \texttt{vehicleRoutes.js}, \texttt{maintenanceRoutes.js}).
    \item \textbf{Controllers (`/controllers`):} Contains the business logic for each route. For example, \texttt{authController.js} handles registration and login, while \texttt{maintenanceController.js} contains the logic to calculate and sort vehicle alerts.
    \item \textbf{Models (`/models`):} Defines the data schemas for MongoDB using Mongoose. The core models are \texttt{User.js}, \texttt{Vehicle.js}, and \texttt{Trip.js}.
    \item \textbf{Middleware (`/middleware`):} Contains functions that process requests before they reach the controller. \texttt{authMiddleware.js} is crucial for protecting routes by verifying the user's JWT.
\end{itemize}

\subsection{Data model}
The NoSQL data model, implemented in MongoDB, was chosen for its flexibility. The primary collections are \texttt{Users}, \texttt{Vehicles}, and \texttt{Trips}. Relationships are managed through object references (e.g., a \texttt{Trip} document contains a reference to the \texttt{\_id} of the \texttt{User} and \texttt{Vehicle}). This structure is well-suited for the application's needs, as vehicle attributes and maintenance requirements can vary significantly. A diagram of the data model is presented in Figure \ref{fig:data-model}.

\begin{figure}[h!]
    \centering
    \begin{tikzpicture}[
        node distance=2.5cm and 4cm,
        entity/.style={
            rectangle, rounded corners, draw=black, very thick,
            minimum width=3.5cm, minimum height=1.5cm, align=left,
            text width=4.5cm, fill=blue!10, drop shadow
        },
        arrow/.style={-{Latex[length=3mm]}, very thick}
    ]

    % Entities
    \node[entity] (user) {\textbf{User Collection} \\ \texttt{\_id} (PK) \\ \texttt{name: String} \\ \texttt{email: String} \\ \texttt{achievements: [String]}};
    \node[entity, right=of user] (vehicle) {\textbf{Vehicle Collection} \\ \texttt{\_id} (PK) \\ \texttt{owner: ObjectId} (FK) \\ \texttt{make: String} \\ \texttt{emissions: Number}};
    \node[entity, below=of vehicle] (trip) {\textbf{Trip Collection} \\ \texttt{\_id} (PK) \\ \texttt{driver: ObjectId} (FK) \\ \texttt{vehicle: ObjectId} (FK) \\ \texttt{distance: Number}};

    % Relationships
    \draw[arrow] (user.east) -- (vehicle.west) node[midway, above, text width=2cm, align=center] {owns};
    \draw[arrow] (vehicle.south) -- (trip.north) node[midway, right] {used in};
    \path (user.south) edge[arrow, bend right=30] node[midway, left] {logs} (trip.west);

    \end{tikzpicture}
    \caption{Simplified Data Model illustrating the relationships between the primary collections.}
    \label{fig:data-model}
\end{figure}

\section{Data flow and API design}
Communication between the frontend and backend occurs over HTTPS via a RESTful API using standard HTTP methods and JSON. A typical data flow for a core feature, such as getting a vehicle recommendation, is as follows:
\begin{enumerate}
    \item The user enters a trip distance on the "Routes" screen and requests a recommendation.
    \item The React Native client sends a POST request to the backend endpoint \texttt{/api/recommendations}. The browser automatically includes the JWT cookie with the request.
    \item The backend's authentication middleware intercepts the request and verifies the JWT.
    \item The recommendation controller fetches all vehicles associated with that user from the database.
    \item The controller's algorithm calculates the estimated emissions for each vehicle and responds with a sorted JSON array.
    \item The frontend parses the response and displays the recommendations to the user.
\end{enumerate}
A complete specification of all API endpoints is available in Appendix A.

\section{Non-functional requirements}
Beyond the core features, the system was designed with several non-functional requirements in mind:
\begin{itemize}
    \item \textbf{Cross-platform compatibility:} React Native and Expo ensure a single codebase can be deployed to both iOS and Android, as well as the web.
    \item \textbf{Scalability:} The backend is stateless, allowing for horizontal scaling. MongoDB also offers robust scaling capabilities.
    \item \textbf{Security:} User authentication is handled using JWTs stored in secure, HTTP-only cookies. Passwords are never stored in plain text, only as bcrypt hashes.
    \item \textbf{Maintainability:} The project is structured into logical modules with a clear separation of concerns. This modularity, combined with TypeScript, improves code quality and makes future maintenance easier.
    \item \textbf{Usability:} The application provides internationalization support and aims for a clean, intuitive user interface.
\end{itemize}

\section{Technology choices justification}
\begin{itemize}
    \item \textbf{React Native (with Expo):} Chosen for its ability to build native-quality applications from a single JavaScript/TypeScript codebase. Expo further accelerates development by simplifying the build process and managing native project configuration.
    \item \textbf{Node.js with Express.js:} Selected for the backend due to its high performance with asynchronous I/O, making it ideal for an API-driven application. Its use of JavaScript creates language synergy with the frontend.
    \item \textbf{MongoDB:} A NoSQL database was preferred over a relational one due to its flexible schema, which is advantageous for storing vehicle data where attributes and maintenance items can vary.
    \item \textbf{Docker:} The inclusion of a \texttt{docker-compose.yaml} file facilitates a consistent and isolated development environment, eliminating "it works on my machine" issues.
    \item \textbf{Jest \& Playwright:} Selected for testing. Jest is a standard for unit testing Node.js applications, while Playwright is a powerful tool for E2E testing that simulates real user interactions.
\end{itemize}

\section{Gamification and sustainability logic}
The logic for these features is implemented on the backend to ensure consistency and security.
\begin{itemize}
    \item \textbf{Sustainability recommendation:} The algorithm calculates total CO\textsubscript{2} emissions for a trip using the formula:
    \begin{equation}
        \text{Emissions (gCO$_2$)} = \text{Distance (km)} \times \text{Emission Factor (gCO$_2$/km)}
    \end{equation}
    The system recommends the vehicle with the lowest calculated emissions.
    \item \textbf{Gamification:} The backend tracks user actions to award achievements. For example, upon creating the first vehicle, the 'FIRST\_VEHICLE' key is added to the user's \texttt{achievements} array in the database. When a trip is logged that surpasses a distance milestone (e.g., 1000 km), the corresponding achievement key is granted.
\end{itemize}
