% License
\newgeometry{top=2.5cm, bottom=2.5cm, left=3cm, right=3cm}
\vspace*{\fill}

\begin{figure}[H]
    \includegraphics[width=2cm]{images/prefaces/by-nc-sa.png} Juan Manuel Segura Duarte, 2025
\end{figure}

\copyright 2025 by Juan Manuel Segura Duarte, supervised by Rosana Montes Soldado:

``\textit{Design and development of a sustainability-focused app for optimizing personal vehicle usage}'' \\

This work is licensed under a Creative Commons Attribution-NonCommercial-ShareAlike 4.0 International License (CC BY-NC-SA 4.0).

\begin{tcolorbox}[
    colback=white,
    boxrule=0.5pt,
    fontupper=\scriptsize,
    before upper={\parindent0pt\setlength{\parskip}{2pt}}, % Reduced paragraph spacing
    after upper={\vspace{-3mm}}, % Reduce space after box content
    width=\textwidth,
    top=7pt,
    bottom=7pt,
    left=4pt,
    right=4pt,
]
\section*{\scriptsize\bfseries You are free to:}
\vspace{-\baselineskip} % Removes one line of vertical space
\begin{itemize}[itemsep=1pt,topsep=2pt,leftmargin=18pt,partopsep=0pt]
    \item \textbf{Share} — copy and redistribute the material in any medium or format
    \item \textbf{Adapt} — remix, transform, and build upon the material
\end{itemize}

The licensor cannot revoke these freedoms as long as you follow the license terms.

\vspace{-4mm} % Reduce space between sections
\section*{\scriptsize\bfseries Under the following terms:}
\vspace{-\baselineskip} % Removes one line of vertical space
\begin{itemize}[itemsep=1pt,topsep=2pt,leftmargin=18pt,partopsep=0pt]
    \item \textbf{Attribution} — You must give \href{https://creativecommons.org/licenses/by/4.0/#ref-appropriate-credit}{appropriate credit}, provide a link to the license, and \href{https://creativecommons.org/licenses/by/4.0/#ref-indicate-changes}{indicate if changes were made}. You may do so in any reasonable manner, but not in any way that suggests the licensor endorses you or your use.
    \item \textbf{NonCommercial} — You may not use the material for commercial purposes.
    \item \textbf{ShareAlike} — If you remix, transform, or build upon the material, you must distribute your contributions under the same license as the original.
    \item \textbf{No additional restrictions} — You may not apply legal terms or \href{https://creativecommons.org/licenses/by/4.0/#ref-technological-measures}{technological measures} that legally restrict others from doing anything the license permits.
\end{itemize}

\vspace{-4mm} % Reduce space between sections
\section*{\scriptsize\bfseries Notices:}
\vspace{-\baselineskip} % Removes one line of vertical space
\begin{itemize}[itemsep=1pt,leftmargin=18pt,partopsep=0pt]
    \item You do not have to comply with the license for elements of the material in the public domain or where your use is permitted by an applicable \href{https://creativecommons.org/licenses/by/4.0/#ref-exception-or-limitation}{exception or limitation}.
    \item No warranties are given. The license may not give you all of the permissions necessary for your intended use. For example, other rights such as \href{https://creativecommons.org/licenses/by/4.0/#ref-publicity-privacy-or-moral-rights}{publicity, privacy, or moral rights} may limit how you use the material.
\end{itemize}
\end{tcolorbox}
\restoregeometry

% % Empty page
% \clearpage
% \mbox{}
% \newpage
% Pagebreak
\newpage


% % Empty page
% \clearpage
% \mbox{}
% \newpage
% Pagebreak
\newpage


% Abstract in English
\begin{center}
    {\large\bfseries Design and development of a sustainability-focused app for optimizing personal vehicle usage}
\end{center}
\begin{center}
    Juan Manuel Segura Duarte
\end{center}

\begin{flushleft}
    \textbf{Keywords:} Sustainable Mobility, Vehicle Fleet Management, Collaborative Systems, Persuasive Technology, Green Computing, Vehicle Recommendation, Proof-of-Concept, Software Architecture, Mobile Application.
\end{flushleft}

\begin{flushleft}
    \textbf{Abstract:}
\end{flushleft}

The widespread ownership of multiple vehicles within a single household represents a significant, yet largely unaddressed, source of economic and environmental inefficiency. The ad-hoc management of these personal "micro-fleets" leads to coordination friction, neglected maintenance, and suboptimal vehicle selection, contributing to excess fuel consumption and greenhouse gas emissions. This thesis addresses this problem by proposing, designing, and implementing \textit{AlDiaCAR}\footnote{The complete source code for this project is available at: \url{https://github.com/jseg380/Trabajo-Fin-Grado}}, a novel, collaborative mobile application conceived as a "Digital Garage" for the modern family.

\textgap

The system is architected upon a four-pillar framework designed to holistically tackle the core pain points: Coordination, Vehicle Recommendation, Automated Maintenance, and Analytics. The intellectual centerpiece of this research is the "Proactive Co-Pilot," a multi-factor recommendation engine that acts as a persuasive technology, nudging users toward more sustainable mobility choices by analyzing journey constraints such as Low-Emission Zone (ZBE) regulations.

\textgap

This work provides a comprehensive architectural blueprint for a full-stack, cross-platform solution, utilizing a technology stack of React Native, Node.js, and MongoDB. The viability of the system's most innovative component was successfully validated through the implementation of a proof-of-concept prototype. This functional artifact demonstrates the core ZBE-aware recommendation logic, confirming the technical feasibility of the proposed solution and establishing a robust foundation for future development towards a feature-complete system. The project's primary contribution is, therefore, a validated design and a functional prototype for a system that synthesizes personal fleet management, collaborative tools, and sustainable decision-making into a single, cohesive platform.


% % Empty page
% \clearpage
% \mbox{}
% \newpage
% Pagebreak
\newpage


% Abstract in Spanish
\begin{center}
    {\large\bfseries Diseño y desarrollo de una aplicación para la optimización del uso del vehículo privado con enfoque en sostenibilidad}
\end{center}
\begin{center}
    Juan Manuel Segura Duarte
\end{center}

\begin{flushleft}
    \textbf{Palabras clave:} Movilidad Sostenible, Gestión de Flotas de Vehículos, Sistemas Colaborativos, Tecnología Persuasiva, Green IT, Recomendación de Vehículos, Prueba de Concepto, Arquitectura de Software, Aplicación Móvil.
\end{flushleft}

\begin{flushleft}
    \textbf{Resumen:}
\end{flushleft}

La propiedad generalizada de múltiples vehículos dentro de un mismo hogar representa una fuente significativa, y en gran medida desatendida, de ineficiencia económica y medioambiental. La gestión improvisada de estas "microflotas" personales conduce a fricciones de coordinación, un mantenimiento deficiente y una selección de vehículos subóptima, contribuyendo a un exceso de consumo de combustible y a la emisión de gases de efecto invernadero. Esta tesis aborda este problema mediante la propuesta, el diseño y la implementación de \textit{AlDiaCAR}\footnote{El código fuente completo de este proyecto está disponible en: \url{https://github.com/jseg380/Trabajo-Fin-Grado}}, una novedosa aplicación móvil colaborativa concebida como un "Garaje Digital" inteligente para la familia moderna.

\textgap

El sistema se articula sobre una arquitectura de cuatro pilares, diseñada para abordar de manera integral los principales puntos débiles identificados: Coordinación, Recomendación Inteligente, Mantenimiento Automatizado y Analíticas. El eje intelectual de esta investigación es el "Copiloto Proactivo", un motor de recomendación multifactorial que actúa como una tecnología persuasiva, incentivando a los usuarios a tomar decisiones de movilidad más sostenibles mediante el análisis de restricciones del trayecto, como las regulaciones de las Zonas de Bajas Emisiones (ZBE).

\textgap

Este trabajo aporta un plan arquitectónico completo para una solución \textit{full-stack} y multiplataforma, utilizando un conjunto de tecnologías compuesto por React Native, Node.js y MongoDB. La viabilidad del componente más innovador del sistema fue validada con éxito a través de la implementación de una prueba de concepto. Este prototipo funcional demuestra la lógica central de recomendación consciente de las ZBE, confirmando la factibilidad técnica de la solución propuesta y estableciendo una base robusta para el futuro desarrollo hacia un sistema con todas las funcionalidades. La contribución principal del proyecto es, por tanto, un diseño validado y un prototipo funcional para un sistema que sintetiza la gestión de flotas personales, las herramientas colaborativas y la toma de decisiones sostenibles en una única plataforma cohesionada.

% % Empty page
% \clearpage
% \mbox{}
% \newpage
% Pagebreak
\newpage


\begin{center}
    \large\bfseries \textit{Acknowledgements}
\end{center}

I would like to take this opportunity to express my heartfelt gratitude to all those who have supported me throughout my journey in completing this thesis.

\textgap

First and foremost, I would like to thank my family for their unwavering love and encouragement. Your belief in my abilities has been a constant source of motivation, and I am forever grateful for your sacrifices and support.

\textgap

I am also grateful to my friends and peers who have stood by me during this challenging process. Your camaraderie, late-night study sessions, and shared laughter have made this journey not only bearable but also enjoyable. Thank you for being my sounding board and for always believing in me.

\textgap

Additionally, I would like to acknowledge the faculty and staff of the University of Granada for providing a nurturing academic environment. Your dedication to teaching and mentorship has profoundly impacted my educational experience.

\textgap

This thesis is not just a reflection of my efforts but a testament to the collective support of all those around me. Thank you for being part of this journey.

% % Empty page
% \clearpage
% \mbox{}
% \newpage
% Pagebreak
\newpage
