% License
\newgeometry{top=2.5cm, bottom=2.5cm, left=3cm, right=3cm}
\vspace*{\fill}

\begin{figure}[H]
    \includegraphics[width=2cm]{images/by-nc-sa.png} Juan Manuel Segura Duarte, 2025
\end{figure}

\copyright 2025 by Juan Manuel Segura Duarte, supervised by Rosa Ana Montes Soldado:

``\textit{Design and development of a sustainability-focused app for optimizing personal vehicle usage}'' \\

This work is licensed under a Creative Commons Attribution-NonCommercial-ShareAlike 4.0 International License (CC BY-NC-SA 4.0).

\begin{tcolorbox}[
    colback=white,
    boxrule=0.5pt,
    fontupper=\scriptsize,
    before upper={\parindent0pt\setlength{\parskip}{2pt}}, % Reduced paragraph spacing
    after upper={\vspace{-3mm}}, % Reduce space after box content
    width=\textwidth,
    top=7pt,
    bottom=7pt,
    left=4pt,
    right=4pt,
]
\section*{\scriptsize\bfseries You are free to:}
\vspace{-\baselineskip} % Removes one line of vertical space
\begin{itemize}[itemsep=1pt,topsep=2pt,leftmargin=18pt,partopsep=0pt]
    \item \textbf{Share} — copy and redistribute the material in any medium or format
    \item \textbf{Adapt} — remix, transform, and build upon the material
\end{itemize}

The licensor cannot revoke these freedoms as long as you follow the license terms.

\vspace{-4mm} % Reduce space between sections
\section*{\scriptsize\bfseries Under the following terms:}
\vspace{-\baselineskip} % Removes one line of vertical space
\begin{itemize}[itemsep=1pt,topsep=2pt,leftmargin=18pt,partopsep=0pt]
    \item \textbf{Attribution} — You must give \href{https://creativecommons.org/licenses/by/3.0/#ref-appropriate-credit}{appropriate credit}, provide a link to the license, and \href{https://creativecommons.org/licenses/by/3.0/#ref-indicate-changes}{indicate if changes were made}. You may do so in any reasonable manner, but not in any way that suggests the licensor endorses you or your use.
    \item \textbf{NonCommercial} — You may not use the material for commercial purposes.
    \item \textbf{ShareAlike} — If you remix, transform, or build upon the material, you must distribute your contributions under the same license as the original.
    \item \textbf{No additional restrictions} — You may not apply legal terms or \href{https://creativecommons.org/licenses/by/3.0/#ref-technological-measures}{technological measures} that legally restrict others from doing anything the license permits.
\end{itemize}

\vspace{-4mm} % Reduce space between sections
\section*{\scriptsize\bfseries Notices:}
\vspace{-\baselineskip} % Removes one line of vertical space
\begin{itemize}[itemsep=1pt,leftmargin=18pt,partopsep=0pt]
    \item You do not have to comply with the license for elements of the material in the public domain or where your use is permitted by an applicable \href{https://creativecommons.org/licenses/by/3.0/#ref-exception-or-limitation}{exception or limitation}.
    \item No warranties are given. The license may not give you all of the permissions necessary for your intended use. For example, other rights such as \href{https://creativecommons.org/licenses/by/3.0/#ref-publicity-privacy-or-moral-rights}{publicity, privacy, or moral rights} may limit how you use the material.
\end{itemize}
\end{tcolorbox}
\restoregeometry

% Empty page
\clearpage
\mbox{}
\newpage


% Empty page
\clearpage
\mbox{}
\newpage


% Abstract in English
\begin{center}
    {\large\bfseries Design and development of a sustainability-focused app for optimizing personal vehicle usage}
\end{center}
\begin{center}
    Juan Manuel Segura Duarte
\end{center}

\begin{flushleft}
    \textbf{Keywords:}
\end{flushleft}

\begin{flushleft}
    \textbf{Abstract:}
\end{flushleft}


This project presents the design and development of \textit{AlDiaCAR}, a cross-platform mobile and web application aimed at promoting sustainable and conscious private vehicle use within households. The application is designed to support families or shared-living environments where multiple drivers manage one or more vehicles, offering tools to coordinate usage, track maintenance, and encourage environmentally-aware decisions.

\textit{AlDiaCAR} provides three core functionalities: (1) vehicle maintenance tracking with smart reminders based on real usage data, (2) driver profiling and usage history to support informed coordination, and (3) a recommendation system that suggests the most sustainable vehicle for a given route based on estimated emissions and current vehicle conditions. To further engage users, the application incorporates gamification and statistical feedback, fostering eco-responsible habits.

\textgap

The system architecture is built with Expo, a React Native framework, for the frontend and Express.js, a Node.js framework, for the backend, supported by a NoSQL database, MongoDB, for flexible, scalable data management. The project follows modular design principles to ensure extensibility and potential integration with external APIs.

\textgap

This report documents the design rationale, development process, and future directions of the application, which aligns with the broader goals of sustainable software and green computing initiatives.


% Empty page
\clearpage
\mbox{}
\newpage


% Abstract in Spanish
\begin{center}
    {\large\bfseries Diseño y desarrollo de una aplicación para la optimización del uso del vehículo privado con enfoque en sostenibilidad}
\end{center}
\begin{center}
    Juan Manuel Segura Duarte
\end{center}

\begin{flushleft}
    \textbf{Palabras clave:}
\end{flushleft}

\begin{flushleft}
    \textbf{Resumen:}
\end{flushleft}

Este proyecto presenta el diseño y desarrollo de \textit{AlDiaCAR}, una aplicación multiplataforma (web y móvil) orientada a fomentar un uso sostenible y consciente del vehículo privado dentro del entorno familiar. La aplicación está diseñada para apoyar a familias o entornos de convivencia compartida donde varios conductores gestionan uno o más vehículos, ofreciendo herramientas para coordinar su uso, realizar un seguimiento del mantenimiento y fomentar decisiones respetuosas con el medio ambiente.

\textgap

\textit{AlDiaCAR} ofrece tres funcionalidades principales: (1) seguimiento del mantenimiento del vehículo con recordatorios inteligentes basados en datos reales de uso, (2) perfiles de conductor e historial de uso para facilitar una coordinación informada, y (3) un sistema de recomendaciones que sugiere el vehículo más sostenible para una ruta determinada en función de las emisiones estimadas y del estado actual del vehículo. Para implicar aún más a los usuarios, la aplicación incorpora elementos de gamificación y retroalimentación estadística que fomentan hábitos ecológicos y responsables.

\textgap

La arquitectura del sistema se ha desarrollado con Expo, un framework de React Native, para el frontend y Express.js, un framework de Node.js, para el backend, utilizando una base de datos NoSQL, MongoDB, como sistema de almacenamiento de datos flexible y escalable. El proyecto sigue principios de diseño modular para garantizar su extensibilidad y la posible integración con APIs externas.

\textgap

% Empty page
\clearpage
\mbox{}
\newpage


\begin{flushleft}
    Professor \textbf{Rosa Ana Montes Soldado}, University of Granada
\end{flushleft}


\textgap

To Whom It May Concern:

\textgap

I hereby certify that the thesis entitled ``\textit{Design and development of a sustainability-focused app for optimizing personal vehicle usage}'' has been prepared under my supervision by Juan Manuel Segura Duarte and authorize its defense before the designated examination committee.

\textgap

For the record, this document is issued and signed in Granada, June 2025.

\vspace{1cm}

Thesis Advisor:

\vspace{5cm}

\noindent Rosa Ana Montes Soldado


% Empty page
\clearpage
\mbox{}
\newpage


\begin{center}
    \large\bfseries \textit{Acknowledgements}
\end{center}

I would like to take this opportunity to express my heartfelt gratitude to all those who have supported me throughout my journey in completing this thesis.

\textgap

First and foremost, I would like to thank my family for their unwavering love and encouragement. Your belief in my abilities has been a constant source of motivation, and I am forever grateful for your sacrifices and support.

\textgap

I am also grateful to my friends and peers who have stood by me during this challenging process. Your camaraderie, late-night study sessions, and shared laughter have made this journey not only bearable but also enjoyable. Thank you for being my sounding board and for always believing in me.

\textgap

Additionally, I would like to acknowledge the faculty and staff of the University of Granada for providing a nurturing academic environment. Your dedication to teaching and mentorship has profoundly impacted my educational experience.

\textgap

This thesis is not just a reflection of my efforts but a testament to the collective support of all those around me. Thank you for being part of this journey.

% Empty page
\clearpage
\mbox{}
\newpage
