% License
\newgeometry{top=2.5cm, bottom=2.5cm, left=3cm, right=3cm}
\vspace*{\fill}

\begin{figure}[H]
    \includegraphics[width=2cm]{images/prefaces/by-nc-sa.png} Juan Manuel Segura Duarte, 2025
\end{figure}

\copyright 2025 by Juan Manuel Segura Duarte, supervised by Rosana Montes Soldado:

``\textit{Design and development of a sustainability-focused app for optimizing personal vehicle usage}'' \\

This work is licensed under a Creative Commons Attribution-NonCommercial-ShareAlike 4.0 International License (CC BY-NC-SA 4.0).

\begin{tcolorbox}[
    colback=white,
    boxrule=0.5pt,
    fontupper=\scriptsize,
    before upper={\parindent0pt\setlength{\parskip}{2pt}}, % Reduced paragraph spacing
    after upper={\vspace{-3mm}}, % Reduce space after box content
    width=\textwidth,
    top=7pt,
    bottom=7pt,
    left=4pt,
    right=4pt,
]
\section*{\scriptsize\bfseries You are free to:}
\vspace{-\baselineskip} % Removes one line of vertical space
\begin{itemize}[itemsep=1pt,topsep=2pt,leftmargin=18pt,partopsep=0pt]
    \item \textbf{Share} — copy and redistribute the material in any medium or format
    \item \textbf{Adapt} — remix, transform, and build upon the material
\end{itemize}

The licensor cannot revoke these freedoms as long as you follow the license terms.

\vspace{-4mm} % Reduce space between sections
\section*{\scriptsize\bfseries Under the following terms:}
\vspace{-\baselineskip} % Removes one line of vertical space
\begin{itemize}[itemsep=1pt,topsep=2pt,leftmargin=18pt,partopsep=0pt]
    \item \textbf{Attribution} — You must give \href{https://creativecommons.org/licenses/by/4.0/#ref-appropriate-credit}{appropriate credit}, provide a link to the license, and \href{https://creativecommons.org/licenses/by/4.0/#ref-indicate-changes}{indicate if changes were made}. You may do so in any reasonable manner, but not in any way that suggests the licensor endorses you or your use.
    \item \textbf{NonCommercial} — You may not use the material for commercial purposes.
    \item \textbf{ShareAlike} — If you remix, transform, or build upon the material, you must distribute your contributions under the same license as the original.
    \item \textbf{No additional restrictions} — You may not apply legal terms or \href{https://creativecommons.org/licenses/by/4.0/#ref-technological-measures}{technological measures} that legally restrict others from doing anything the license permits.
\end{itemize}

\vspace{-4mm} % Reduce space between sections
\section*{\scriptsize\bfseries Notices:}
\vspace{-\baselineskip} % Removes one line of vertical space
\begin{itemize}[itemsep=1pt,leftmargin=18pt,partopsep=0pt]
    \item You do not have to comply with the license for elements of the material in the public domain or where your use is permitted by an applicable \href{https://creativecommons.org/licenses/by/4.0/#ref-exception-or-limitation}{exception or limitation}.
    \item No warranties are given. The license may not give you all of the permissions necessary for your intended use. For example, other rights such as \href{https://creativecommons.org/licenses/by/4.0/#ref-publicity-privacy-or-moral-rights}{publicity, privacy, or moral rights} may limit how you use the material.
\end{itemize}
\end{tcolorbox}
\restoregeometry

% % Empty page
% \clearpage
% \mbox{}
% \newpage
% Pagebreak
\newpage


% % Empty page
% \clearpage
% \mbox{}
% \newpage
% Pagebreak
\newpage


% Abstract in English
\begin{center}
    {\large\bfseries Design and development of a sustainability-focused app for optimizing personal vehicle usage}
\end{center}
\begin{center}
    Juan Manuel Segura Duarte
\end{center}

\begin{flushleft}
    \textbf{Keywords:} Sustainable Mobility, Green IT, Full-Stack Development, React Native, Vehicle Management, Gamification.
\end{flushleft}

\begin{flushleft}
    \textbf{Abstract:}
\end{flushleft}

This project presents the design and development of \textit{AlDiaCAR}\footnote{The complete source code for this project is available at: \url{https://github.com/jseg380/Trabajo-Fin-Grado}}, a cross-platform mobile and web application aimed at promoting sustainable and conscious use of a personal fleet of vehicles. The application is designed to support an individual user managing one or more vehicles, offering tools to centralize maintenance tracking and encourage environmentally-aware decisions.

\textit{AlDiaCAR} provides three core functionalities: (1) unified vehicle maintenance tracking with smart reminders based on upcoming dates and distance-based intervals, (2) a trip logging system that updates vehicle metrics and user statistics, and (3) a recommendation system that suggests the most sustainable vehicle for a given route based on estimated emissions. To further engage the user, the application incorporates gamification elements and statistical feedback, fostering eco-responsible habits.

\textgap

The system architecture is built with Expo (a React Native framework) for the frontend and Express.js (a Node.js framework) for the backend, supported by a MongoDB database for flexible data management. The project was developed following modular design principles and includes a comprehensive testing suite to ensure quality and reliability.

\textgap

This report documents the design rationale, implementation details, and validation of the application, which aligns with the broader goals of sustainable software and "Green through IT" initiatives.

% % Empty page
% \clearpage
% \mbox{}
% \newpage
% Pagebreak
\newpage


% Abstract in Spanish
\begin{center}
    {\large\bfseries Diseño y desarrollo de una aplicación para la optimización del uso del vehículo privado con enfoque en sostenibilidad}
\end{center}
\begin{center}
    Juan Manuel Segura Duarte
\end{center}

\begin{flushleft}
    \textbf{Palabras clave:} Movilidad Sostenible, TI Verde, Desarrollo Full-Stack, React Native, Gestión de Vehículos, Gamificación.
\end{flushleft}

\begin{flushleft}
    \textbf{Resumen:}
\end{flushleft}

Este proyecto presenta el diseño y desarrollo de \textit{AlDiaCAR}\footnote{El código fuente completo de este proyecto está disponible en: \url{https://github.com/jseg380/Trabajo-Fin-Grado}}, una aplicación multiplataforma (móvil y web) orientada a fomentar un uso sostenible y consciente de una flota de vehículos personal. La aplicación está diseñada para dar soporte a un usuario individual que gestiona uno o más vehículos, ofreciendo herramientas para centralizar el seguimiento del mantenimiento y fomentar decisiones respetuosas con el medio ambiente.

\textit{AlDiaCAR} ofrece tres funcionalidades principales: (1) seguimiento unificado del mantenimiento del vehículo con recordatorios inteligentes basados en fechas próximas e intervalos de distancia, (2) un sistema de registro de viajes que actualiza las métricas del vehículo y las estadísticas del usuario, y (3) un sistema de recomendaciones que sugiere el vehículo más sostenible para una ruta determinada en función de las emisiones estimadas. Para implicar al usuario, la aplicación incorpora elementos de gamificación y retroalimentación estadística que fomentan hábitos eco-responsables.

\textgap

La arquitectura del sistema se ha desarrollado con Expo (un framework de React Native) para el frontend y Express.js (un framework de Node.js) para el backend, utilizando una base de datos MongoDB para una gestión de datos flexible. El proyecto se ha desarrollado siguiendo principios de diseño modular e incluye una completa suite de tests para garantizar su calidad y fiabilidad.

\textgap

Este informe documenta la justificación del diseño, los detalles de implementación y la validación de la aplicación, que se alinea con los objetivos más amplios del software sostenible y las iniciativas de "TI Verde" (Green through IT).

% % Empty page
% \clearpage
% \mbox{}
% \newpage
% Pagebreak
\newpage


\begin{center}
    \large\bfseries \textit{Acknowledgements}
\end{center}

I would like to take this opportunity to express my heartfelt gratitude to all those who have supported me throughout my journey in completing this thesis.

\textgap

First and foremost, I would like to thank my family for their unwavering love and encouragement. Your belief in my abilities has been a constant source of motivation, and I am forever grateful for your sacrifices and support.

\textgap

I am also grateful to my friends and peers who have stood by me during this challenging process. Your camaraderie, late-night study sessions, and shared laughter have made this journey not only bearable but also enjoyable. Thank you for being my sounding board and for always believing in me.

\textgap

Additionally, I would like to acknowledge the faculty and staff of the University of Granada for providing a nurturing academic environment. Your dedication to teaching and mentorship has profoundly impacted my educational experience.

\textgap

This thesis is not just a reflection of my efforts but a testament to the collective support of all those around me. Thank you for being part of this journey.

% % Empty page
% \clearpage
% \mbox{}
% \newpage
% Pagebreak
\newpage
