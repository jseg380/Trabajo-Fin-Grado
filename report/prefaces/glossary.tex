\makeglossaries

\newglossaryentry{autonomo}{
    name=Autónomo,
    text=autónomo,
    description={An autónomo is the Spanish legal and fiscal term for a self-employed individual or sole trader, roughly equivalent to a "freelancer" or "independent contractor" in anglophone contexts. This is a highly prevalent form of employment in Spain, encompassing a vast range of professions from skilled tradespeople like plumbers and electricians to professional service providers such as consultants, designers, and lawyers. Unlike employees with fixed 9-to-5 schedules, autónomos typically manage their own working hours and client engagements, resulting in a highly variable and often unpredictable daily schedule. This lack of a fixed routine is a critical factor in the user persona for this thesis, as it complicates household resource planning—particularly the availability and use of shared vehicles—necessitating a dynamic, real-time coordination tool rather than a static, schedule-based system.}
}

\newglossaryentry{API}{
    name=API (Application Programming Interface),
    description={A set of rules and protocols that allows different software applications to communicate. In this project, it refers to the REST API that connects the frontend mobile app to the backend server.}
}

\newglossaryentry{bcryptjs}{
    name=bcryptjs,
    description={A password-hashing function designed to be slow and computationally intensive, which protects against brute-force search attacks. This library is used in the project to securely hash and store user passwords.}
}

\newglossaryentry{CI}{
    name=CI/CD (Continuous Integration/Continuous Deployment),
    description={A software development practice where developers regularly merge their code changes into a central repository, after which automated builds and tests are run. In this project, CI is implemented via GitHub Actions to automatically run backend and frontend tests on every pull request.}
}

\newglossaryentry{CORS}{
    name=CORS (Cross-Origin Resource Sharing),
    description={A browser security feature that restricts web pages from making requests to a different domain than the one that served the page. The backend uses the \texttt{cors} middleware to explicitly allow the frontend (e.g., at \texttt{localhost:3000}) to access the API (at \texttt{localhost:5000}).}
}

\newglossaryentry{CRUD}{
    name=CRUD,
    description={An acronym for Create, Read, Update, and Delete, which are the four basic functions of persistent storage. These operations are the foundation of the project's vehicle management API.}
}

\newglossaryentry{distintivo-ambiental}{
    name=Distintivo Ambiental,
    text=Distintivo Ambiental,
    description={The Distintivo Ambiental is an official vehicle classification system implemented by Spain's Dirección General de Tráfico (DGT). It categorizes vehicles based on their pollutant emission levels, assigning a corresponding colored sticker that must be displayed on the vehicle. This system is the primary mechanism used by municipalities to regulate traffic in Low Emission Zones (ZBEs). The badges are not a measure of CO\textsubscript{2} emissions but of local air pollutants like NOx and particulate matter. The main classifications for passenger cars are:
        \newline \textbullet\ \textbf{0 Emisiones (Blue Badge)}: Reserved for the cleanest vehicles, primarily Battery Electric Vehicles (BEVs), extended-range electric vehicles (REEVs), plug-in hybrids (PHEVs) with a range of over 40km, and fuel cell vehicles. They enjoy the most privileges, including unrestricted access to all ZBEs.
        \newline \textbullet\ \textbf{ECO (Green and Blue Badge)}: This category includes standard hybrid vehicles (HEVs), plug-in hybrids with a range under 40km, and vehicles powered by compressed natural gas (CNG) or liquefied petroleum gas (LPG). They have significant privileges, though slightly fewer than '0 Emisiones'.
        \newline \textbullet\ \textbf{C (Green Badge)}: For gasoline vehicles registered from 2006 onwards (Euro 4, 5, 6 standards) and diesel vehicles registered from 2014 onwards (Euro 6). This is a very common category for modern internal combustion engine vehicles.
        \newline \textbullet\ \textbf{B (Yellow Badge)}: For gasoline vehicles registered from 2001 to 2005 (Euro 3) and diesel vehicles from 2006 to 2013 (Euro 5). These vehicles face the most significant access restrictions in ZBEs.
        \newline \textbullet\ \textbf{Sin Distintivo (No Badge)}: The oldest and most polluting vehicles, which are typically barred from entering any ZBE.
    }
}

\newglossaryentry{Docker}{
    name=Docker,
    description={A platform for developing and running applications in isolated environments called containers. It is used in this project via Docker Compose to create a consistent and reproducible development environment for the backend and database.}
}

\newglossaryentry{dotenv}{
    name=dotenv,
    description={A zero-dependency module that loads environment variables from a \texttt{.env} file into \texttt{process.env}. It is used in the backend to manage configuration and secrets like database connection strings and JWT keys.}
}

\newglossaryentry{E2E}{
    name=E2E (End-to-End) Testing,
    description={A testing methodology used to verify the workflow of an application from start to finish. This project uses Playwright to conduct E2E tests by simulating real user scenarios in a browser.}
}

\newglossaryentry{Expo}{
    name=Expo,
    description={A framework and platform for universal React applications. It provides a set of tools and services built around React Native that simplify the development and deployment of the project's mobile app.}
}

\newglossaryentry{Expressjs}{
    name=Express.js,
    description={A minimal and flexible Node.js web application framework that provides a robust set of features for building APIs. It is the core framework for the AlDiaCAR backend.}
}

\newglossaryentry{Gamification}{
    name=Gamification,
    description={The application of game-design elements (like achievements and stats) in non-game contexts to engage users and motivate specific behaviors, such as sustainable driving.}
}

\newglossaryentry{GreenIT}{
    name=Green IT,
    description={The practice of environmentally sustainable computing. It encompasses "Green in IT" (making computing itself more efficient) and "Green through IT" (using IT to enable sustainability in other domains), the latter of which is the focus of this project.}
}

\newglossaryentry{HCI}{
    name=HCI (Human-Computer Interaction),
    description={A multidisciplinary field of study focusing on the design of computer technology and the interaction between humans (the users) and computers.}
}

\newglossaryentry{JSON}{
    name=JSON (JavaScript Object Notation),
    description={A lightweight data-interchange format that is the standard for data transfer in this project's REST API.}
}

\newglossaryentry{JWT}{
    name=JWT (JSON Web Token),
    description={A compact, URL-safe means of representing claims to be transferred between two parties. Used in this project for managing user authentication sessions via secure cookies.}
}

\newglossaryentry{Jest}{
    name=Jest,
    description={A JavaScript testing framework with a focus on simplicity. It is used for the unit and integration testing of the backend API.}
}

\newglossaryentry{Middleware}{
    name=Middleware,
    description={In Express.js, these are functions that execute during the lifecycle of a request to the server. Used extensively in this project for authentication (\texttt{authMiddleware}) and file uploads (\texttt{uploadMiddleware}).}
}

\newglossaryentry{MongoDB}{
    name=MongoDB,
    description={A document-oriented NoSQL database program. It was chosen for this project due to its flexible schema, which is ideal for storing the varied data of users and vehicles.}
}

\newglossaryentry{Mongoose}{
    name=Mongoose,
    description={An Object Data Modeling (ODM) library for MongoDB and Node.js. It is used throughout the backend to define schemas, validate data, and manage relationships between documents.}
}

\newglossaryentry{Monorepo}{
    name=Monorepo,
    description={A software development strategy where code for multiple projects is stored in the same repository. This strategy is used to manage the \texttt{frontend}, \texttt{backend}, and \texttt{tests} codebases together.}
}

\newglossaryentry{Nodejs}{
    name=Node.js,
    description={A back-end JavaScript runtime environment that executes JavaScript code outside a web browser. It is the foundation of the project's backend server.}
}

\newglossaryentry{ODM}{
    name=ODM (Object Data Modeling),
    description={A programming technique for converting data between incompatible type systems in object-oriented programming languages. Mongoose is the ODM used in this project to map JavaScript objects to documents in MongoDB.}
}

\newglossaryentry{PaaS}{
    name=PaaS (Platform-as-a-Service),
    description={A category of cloud computing services that provides a platform allowing customers to develop, run, and manage applications without the complexity of building and maintaining the underlying infrastructure.}
}

\newglossaryentry{Playwright}{
    name=Playwright,
    description={A framework for web testing and automation. It is used for the end-to-end testing of the frontend, simulating user journeys across different browsers and devices.}
}

\newglossaryentry{REST}{
    name=REST (Representational State Transfer),
    description={An architectural style for designing networked applications. The project's backend is a RESTful API that uses HTTP requests to manage resources.}
}

\newglossaryentry{ReactNative}{
    name=React Native,
    description={An open-source UI software framework used to develop applications for Android, iOS, and the Web from a single codebase. It is the core technology of the frontend.}
}

\newglossaryentry{low-emissions-zones}{
    name=Zona de Bajas Emisiones (ZBE),
    text=Low Emission Zones,
    description={A \textit{Zona de Bajas Emisiones (ZBE)} is a geographically defined urban area where access for certain polluting vehicles is restricted to improve air quality. Mandated by Spain's Climate Change and Energy Transition Law, all municipalities with over 50,000 inhabitants are required to establish ZBEs. The implementation and severity of these restrictions are at the discretion of the local city council, leading to a fragmented and often confusing regulatory landscape for drivers.
        \newline Access restrictions are enforced based on the vehicle's Distintivo Ambiental. A ZBE may, for example:
        \newline \textbullet\ \textbf{0 Emisiones (Blue Badge)}: Reserved for the cleanest vehicles, primarily Battery Electric Vehicles (BEVs), extended-range electric vehicles (REEVs), plug-in hybrids (PHEVs) with a range of over 40km, and fuel cell vehicles. They enjoy the most privileges, including unrestricted access to all ZBEs.
        \newline \textbullet\ \textbf{ECO (Green and Blue Badge)}: This category includes standard hybrid vehicles (HEVs), plug-in hybrids with a range under 40km, and vehicles powered by compressed natural gas (CNG) or liquefied petroleum gas (LPG). They have significant privileges, though slightly fewer than '0 Emisiones'.
        \newline \textbullet\ \textbf{C (Green Badge)}: For gasoline vehicles registered from 2006 onwards (Euro 4, 5, 6 standards) and diesel vehicles registered from 2014 onwards (Euro 6). This is a very common category for modern internal combustion engine vehicles.
        \newline \textbullet\ \textbf{B (Yellow Badge)}: For gasoline vehicles registered from 2001 to 2005 (Euro 3) and diesel vehicles from 2006 to 2013 (Euro 5). These vehicles face the most significant access restrictions in ZBEs.
        \newline \textbullet\ \textbf{Sin Distintivo (No Badge)}: The oldest and most polluting vehicles, which are typically barred from entering any ZBE.
    }
}
