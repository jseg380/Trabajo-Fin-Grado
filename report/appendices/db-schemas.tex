\chapter{Mongoose Data Schemas}

This appendix provides the complete source code for the Mongoose schemas that define the data architecture of the \textit{AlDiaCAR} application. These schemas enforce data structure and validation at the application layer before data is persisted to the MongoDB database.

\section{Household Schema}
The \texttt{Household} schema is the central entity for enabling collaborative features. It groups users together and uses the \texttt{nanoid} library to generate a unique, human-readable code that users can share to invite others.

\begin{lstlisting}[language=JavaScript, caption={Source code for \texttt{models/Household.js}}, breaklines=true]
import mongoose from 'mongoose';
import { customAlphabet } from 'nanoid';

// Using a dictionary-based alphabet for more memorable codes
const alphabet = 'ABCDEFGHIJKLMNOPQRSTUVWXYZ0123456789';
const nanoid = customAlphabet(alphabet, 8); // Generates an 8-character code like 'A3B5D9F1'

const HouseholdSchema = new mongoose.Schema({
  name: {
    type: String,
    required: true,
  },
  joinCode: {
    type: String,
    required: true,
    unique: true,
    default: () => nanoid(),
  },
  owner: {
    type: mongoose.Schema.Types.ObjectId,
    ref: 'User',
    required: true,
  },
  members: [{
    type: mongoose.Schema.Types.ObjectId,
    ref: 'User'
  }],
}, { timestamps: true });

export default mongoose.model('Household', HouseholdSchema);
\end{lstlisting}

\section{User Schema}
The \texttt{User} schema defines the structure for user accounts. It includes a reference to the \texttt{Household} the user belongs to. A key feature is the \texttt{pre('save')} middleware hook, which automatically hashes the user's password using \texttt{bcryptjs} before it is saved to the database, ensuring that plain-text passwords are never stored.

\begin{lstlisting}[language=JavaScript, caption={Source code for \texttt{models/User.js}}, breaklines=true]
import mongoose from 'mongoose';
import bcrypt from 'bcryptjs';

const UserSchema = new mongoose.Schema(
  {
    name: {
      type: String,
      required: true,
      trim: true,
    },
    email: {
      type: String,
      required: true,
      unique: true,
      lowercase: true,
      trim: true,
    },
    password: {
      type: String,
      required: true,
      select: false, // Never return password in queries
    },
    household: {
      type: mongoose.Schema.Types.ObjectId,
      ref: 'Household',
    },
    avatar: {
      type: String,
      default: 'images/generic-avatar.png',
    },
    stats: {
      distanceTraveled: { type: Number, default: 0 },
      co2Saved: { type: Number, default: 0 },
      totalVehicles: { type: Number, default: 0 },
    },
    // A simple array to store unique achievement keys
    achievements: {
      type: [String],
      default: [],
    },
  },
  {
    timestamps: true,
  }
);

// Password hashing middleware
UserSchema.pre('save', async function (next) {
  if (!this.isModified('password')) return next();

  const salt = await bcrypt.genSalt(10);
  this.password = await bcrypt.hash(this.password, salt);
  next();
});

// Password verification method
UserSchema.methods.matchPassword = async function (enteredPassword) {
  return await bcrypt.compare(enteredPassword, this.password);
};

export default mongoose.model('User', UserSchema);
\end{lstlisting}

\section{Vehicle Schema}
The \texttt{Vehicle} schema is a complex model that defines a vehicle's core attributes, its real-time status, and its maintenance schedule. The \texttt{status} field is a sub-document that tracks whether the vehicle is available, in use, or reserved. The \texttt{upcomingMaintenance} field is an embedded sub-document that defines default and user-configurable service intervals, which are crucial for the maintenance alert system.

\begin{lstlisting}[language=JavaScript, caption={Source code for \texttt{models/Vehicle.js}}, breaklines=true]
import mongoose from 'mongoose';

// Default generic maintenance intervals
const maintenanceSchema = new mongoose.Schema({
  tires: {
    date: {
      type: Date,
      default: () => new Date(new Date().setFullYear(new Date().getFullYear() + 1)), // One year from now
    },
    distance: {
      type: Number,
      default: 40000,
    },
  },
  brakes: {
    distance: {
      type: Number,
      default: 50000,
    },
  },
  oilChange: {
    distance: {
      type: Number,
      default: 30000, // Default to 30,000 km for synthetic oil (moden cars)
    },
  },
  itv: {
    type: Date,
    default: () => new Date(new Date().setFullYear(new Date().getFullYear() + 2)), // Two years from now
  },
});

const VehicleSchema = new mongoose.Schema({
  make: { type: String, required: true },
  model: { type: String, required: true },
  year: { type: Number, required: true },
  fuelType: { type: String, required: true, enum: ['gasoline', 'diesel', 'electric', 'hybrid'] },
  owner: { type: mongoose.Schema.Types.ObjectId, ref: 'User', required: true },
  status: {
    state: {
      type: String,
      enum: ['at_home', 'in_use', 'reserved'],
      default: 'at_home',
    },
    checkedOutBy: {
      type: mongoose.Schema.Types.ObjectId,
      ref: 'User',
      default: null,
    },
    reservedBy: {
      type: mongoose.Schema.Types.ObjectId,
      ref: 'User',
      default: null,
    },
    reservedFrom: {
      type: Date,
      default: null,
    },
    reservedUntil: {
      type: Date,
      default: null,
    },
    lastUpdated: {
      type: Date,
      default: Date.now,
    },
  },
  emissions: Number,
  upcomingMaintenance: maintenanceSchema,
});

export default mongoose.model('Vehicle', VehicleSchema);
\end{lstlisting}

\section{Trip Schema}
The \texttt{Trip} schema is used to log each journey taken by a user. It stores references to the driver and the vehicle used, as well as key data about the trip such as distance and calculated emissions. This collection provides the raw data that feeds the statistics and gamification features of the application.

\begin{lstlisting}[language=JavaScript, caption={Source code for \texttt{models/Trip.js}}, breaklines=true]
import mongoose from 'mongoose';

const TripSchema = new mongoose.Schema({
  vehicle: { type: mongoose.Schema.Types.ObjectId, ref: 'Vehicle', required: true },
  driver: { type: mongoose.Schema.Types.ObjectId, ref: 'User', required: true },

  locations: {
    start: {
      type: {
        type: String,
        enum: ['Point'],
        required: true,
      },
      coordinates: {
        type: [Number],
        required: true,
      },
    },
    end: {
      type: {
        type: String,
        enum: ['Point'],
        required: true,
      },
      coordinates: {
        type: [Number],
        required: true,
      },
    },
  },

  distance: { type: Number, required: true }, // in km
  date: { type: Date, default: Date.now },
  calculatedEmissions: { type: Number, required: true }, // in grams of CO2
});

export default mongoose.model('Trip', TripSchema);
\end{lstlisting}
