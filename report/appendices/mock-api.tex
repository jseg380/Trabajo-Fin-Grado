\chapter{Appendix: Mock Vehicle Specification API}

To facilitate development and testing without relying on a live, and potentially rate-limited or paid, third-party vehicle data service, a mock API was created. This standalone Node.js server, located at \texttt{src/backend/mock-api/mock-car-api.js}, simulates the functionality required for the "Add Vehicle" user workflow. It provides endpoints to fetch lists of car makes and models, and to get technical specifications for a specific vehicle.

This approach provides several key advantages:
\begin{itemize}
    \item \textbf{Decoupling:} The frontend is developed against a stable API contract, regardless of the availability or cost of an external service.
    \item \textbf{Offline Development:} The entire system can be run and tested without an internet connection.
    \item \textbf{Deterministic Testing:} End-to-end tests can rely on predictable data from the mock API, making them more robust and reliable.
\end{itemize}

The mock API runs on a separate port (7500) and provides the following endpoints.

\section{Endpoint: GET /api/makes}
\begin{itemize}
    \item \textbf{Description:} Returns a comprehensive list of all available car makes. This is used to populate a dropdown menu in the "Add Vehicle" screen.
    \item \textbf{Success Response (200 OK):} An array of make objects.
    \begin{verbatim}
[
    { "make_id": 1, "make": "AC" },
    { "make_id": 2, "make": "Acura" },
    ...
]
    \end{verbatim}
\end{itemize}

\section{Endpoint: GET /api/makes/:make/models}
\begin{itemize}
    \item \textbf{Description:} Returns a list of models for a specific car make, identified by its name. This allows for dynamic, dependent dropdown menus in the UI.
    \item \textbf{Example Request URL:}
    \begin{verbatim}
http://localhost:7500/api/makes/Toyota/models
    \end{verbatim}
    \item \textbf{Success Response (200 OK):} An array of model objects corresponding to the given make.
    \begin{verbatim}
[
    { "model_id": 872, "model": "Camry", "make_id": 140 },
    { "model_id": 876, "model": "Corolla", "make_id": 140 },
    ...
]
    \end{verbatim}
    \item \textbf{Error Response (404 Not Found):} If the make name does not exist in the mock data.
\end{itemize}

\section{Endpoint: GET /api/specs}
\begin{itemize}
    \item \textbf{Description:} This is the core endpoint for fetching technical data. It simulates retrieving the CO\textsubscript{2} emission factor based on the user's final vehicle selection.
    \item \textbf{Request Query Parameters:}
    \begin{verbatim}
?make=<string>&model=<string>&year=<string>&fuelType=<string>
    \end{verbatim}
    \item \textbf{Example Request URL:}
    \begin{verbatim}
http://localhost:7500/api/specs?make=Toyota&model=Corolla&year=2021&fuelType=hybrid
    \end{verbatim}
    \item \textbf{Example Success Response (200 OK):} The API uses simple internal logic to return a plausible emission factor. It also simulates a short network delay to feel more realistic.
    \begin{verbatim}
{
    "emissionFactor": 98,
    "source": "OEM Hybrid Data"
}
    \end{verbatim}
    \item \textbf{Error Response (400 Bad Request):} If any of the required query parameters are missing.
\end{itemize}
